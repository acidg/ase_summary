\section{Introduction}
This lecture is about understanding how to build complex (software) systems, which trade-offs there are between quality and the architecture and how to efficiently deliver the developed software to the stakeholders.
For that, different modeling techniques, system analysis and design with quality trade-offs, patterns, guidelines and best-practices are taught and tools for system configuration, integration and deployment are explained.

The lecture is separated into 4 chapters:
\begin{enumerate}
    \item Context of Software Engineering
        \subitem {\small Introduction, characteristics of software systems in different domains, case studies and factors affecting the design of a software system}
    \item From (quality) requirements to system design
        \subitem {\small Software architecture, libraries and frameworks, antipatterns, model-driven engineering, software product line engineering, safety and security, testability}
    \item Software architectures and their trade-offs
        \subitem {\small Distributed systems and middleware, database-, message-, object-, component- and service-oriented architectures}
    \item From source code to physical deployment
        \subitem {\small Historical perspective, version control, continuous integration and deployment, virtual machines and containers, software architectures for the cloud}
\end{enumerate}

\subsection{Characteristics of Software Systems}
Software eningeering can be separated into the technical \& management and the application domain.
The technical \& management domain contain the software system with limiting factors.
First, there are different targets defining those borders: Beside the technical infrastructure which is defining the system's size, cost, quality, time and the functionality target are set limits.
Those factors again are influenced by the process (high control) used to develop the system and the constraints (limited control: money, stakeholders, environmental influences, ...) limiting the resources available during development of the system.

The application domain is influencing the functionality target (e.g. needed functionality, available inputs and outputs, ...) and the constrains (money, management, ...) in the technical \& management domain.

The application domain can be divided into 
\begin{itemize}[topsep=5pt, itemsep=0pt]
    \item \textbf{Embedded} systems \textit{electronic devices, transportation, intelligent homes, ...}
    \item \textbf{Information} systems \textit{logistics, marketing tools, management tools, ...}
    \item \textbf{Cyberphysical} systems (connects embedded and information system) \textit{networks, reading sensors and putting out information, ...}
    \item (\textbf{Scientific software} systems \textit{not covered})
\end{itemize}
